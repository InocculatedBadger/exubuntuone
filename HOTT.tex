% !TEX TS-program = pdflatex
% !TEX encoding = UTF-8 Unicode

% This is a simple template for a LaTeX document using the "article" class.
% See "book", "report", "letter" for other types of document.

\documentclass[12pt]{article} % use larger type; default would be 10pt
\usepackage{color}
\usepackage[utf8]{inputenc} % set input encoding (not needed with XeLaTeX)

%%% Examples of Article customizations
% These packages are optional, depending whether you want the features they provide.
% See the LaTeX Companion or other references for full information.

%%% PAGE DIMENSIONS

\usepackage[top=20mm,right=20mm,bottom=15mm,left=20mm]{geometry}
% \geometry{margins=2in} % for example, change the margins to 2 inches all round
% \geometry{landscape} % set up the page for landscape
%   read geometry.pdf for detailed page layout information

\usepackage{graphicx} % support the \includegraphics command and options

% \usepackage[parfill]{parskip} % Activate to begin paragraphs with an empty line rather than an indent

%%% PACKAGES
\usepackage{booktabs} % for much better looking tables
\usepackage{array} % for better arrays (eg matrices) in maths
%\usepackage{paralist} % very flexible & customisable lists (eg. enumerate/itemize, etc.)

%\usepackage{subfig} % make it possible to include more than one captioned figure/table in a single float
\usepackage{amsfonts}
\usepackage{amsthm}
\usepackage{tikz}
\usepackage{amsmath}
\usepackage{float}
\usepackage{graphicx}
\usepackage{caption}
\usepackage{subcaption}
\usepackage{color}
\usepackage{amssymb}
\usepackage{bm}

% These packages are all incorporated in the memoir class to one degree or another...

%%% HEADERS & FOOTERS
\usepackage{fancyhdr} % This should be set AFTER setting up the page geometry
\pagestyle{fancy} % options: empty , plain , fancy
\renewcommand{\headrulewidth}{0pt} % customise the layout...
\lhead{}\chead{}\rhead{}
\lfoot{}\cfoot{\thepage}\rfoot{}

%%% SECTION TITLE APPEARANCE
%\usepackage{sectsty}
%\allsectionsfont{\sffamily\mdseries\upshape} % (See the fntguide.pdf for font help)
% (This matches ConTeXt defaults)

%%% ToC (table of contents) APPEARANCE
%\usepackage[nottoc,notlof,notlot]{tocbibind} % Put the bibliography in the ToC
%\usepackage[titles,subfigure]{tocloft} % Alter the style of the Table of Contents
%\renewcommand{\cftsecfont}{\rmfamily\mdseries\upshape}
%\renewcommand{\cftsecpagefont}{\rmfamily\mdseries\upshape} % No bold!


\newtheorem{theorem}{Theorem} 
\newtheorem{lemma}{Lemma}
\newtheorem{propn}{Proposition}
\newtheorem*{thmm}{Theorem}
\newtheorem{remk}{Remark} 
\newtheorem{corol}{Corollary}
\newtheorem{definition}{Definition}



\newtheorem{thm}{Theorem}[section] 
\newtheorem{prop}[thm]{Proposition} 
\newtheorem{lem}[thm]{Lemma}
\newtheorem{cor}[thm]{Corollary} 
\newtheorem{con}[thm]{Conjecture} 

\theoremstyle{definition}
\newtheorem{defn}[thm]{Definition}
\newtheorem*{rem}{Remark}
%\newtheorem*{nota}{Notation}
\newtheorem*{nota}{Notation}
\newtheorem{cla}[thm]{Claim}
\newtheorem{ex}[thm]{Example}
\newtheorem{exs}[thm]{Examples}
\newtheorem*{exer}{Exercise}
\newtheorem{case}{Case}
\newtheorem{conj}{Conjecture}

\definecolor{sotonblue}{rgb}{0.0,0.394,0.597}


%opening
 \title{HTT}
\author{David Matthews }

\begin{document}
\section{Introduction}
\section{Path Induction}
\section{Functions are Functors}
% $ap_{f}(refl_{x}) = refl_{f(x)}$  and by lemma \ref{} $refl_{x} \Box refl_{x} = refl_{x}$.
 \section{Chapter 2.3 Type Families are fibrations}
Every proof in the relevant section in the book is proved by path induction.
   

Let $P:A \rightarrow U$ be a type family which we can consider as a fibration. 

Consider some $x,y:A$ and a path $p: x =_{A} y$ this will be our base space. The total space is $\sum_{x:A}P(x)$ which is the dependent pair types.  This means that given any $x:A$ there exists a fibre $P(x): \sum_{x:A}P(x)$ of elements that map to $x:A$ via the first projection $\pi(x,u) = x$.  

\begin{lem}{Transport}\label{lem:2.3.1}
 There exists a function $transport^{P}(p,_) := p_{*}: P(x) \rightarrow P(y)$.
\end{lem}

\begin{proof}
Let $p$ be $refl_{x}$ and use path induction. Note that $(refl_{x})_{*} = id_{P(x)}$.
\end{proof}

Homotopy Version:
Given any two points, $x,y$ in base space $A$ and fibers $P(x)$ and $P(y)$ above $x$ and $y$ respectively then there exists a function from $P(x)$ to $P(y)$ 



\begin{lem}{Path Lifting Property}\label{lem:2.3.2}
 Fix some $u:P(x)$ then there exists a path
 \[lift(u,p): (x,u) = (y,p_{*}(u))\]
\end{lem}

\begin{proof}
 Assume that $p$ is $refl_{x}$.  By the proof of lemma \ref{lem:2.3.1} $(refl_{x})_{*} = id_{P(x)}$, hence;
 \[lift(u,refl_{x}) := ((x,u) = (x,u)) := refl_{(x,u)}\]
 Now path inductions give us the existence of $lift$.  
\end{proof}
So given a path, $p: x=y$, in the base space and and any element of the fibre $P(x)$ there exists a path, $p': (x,u) = (y,p_{*}(u)$ such that $(x,u):P(x)$ the fibre over $x$ and $(y,p_{*}(u):P(y)$ the fibre over $y$.  I.e. we have lifted the path $p$.

Given some $f : \Pi_{x:A}P(x)$ we can define a non-dependent function $f':A \rightarrow \sum_{x:A}P(x)$ by setting $p'(x) :=(x,f(x))$.  In homotopological terms $f'$ is a right continuous inverse to $pi_{1}$ i.e. a section.

\begin{lem}\label{lem:2.3.4}
 Fix $f$ be as above, then there exists a dependent map:
 \[apd_{f} \Pi_{p:x=y}p_{*}(f(x)) = _{P_{y}} f(y)\]
 \end{lem}

 \begin{proof}
  So suppose (as usual) that $p$ is $refl_{x}$.  Then
  \[(refl_{x})_{*}(f(x)) = _{P(y)} f(x)\]
  \[:=refl_{f(x)}\]
  By path induction $apd_{f}$ exists.
 \end{proof}
So, given some section, $f':=(x,f(x)$,  there exists a map over the section $P(y)$ from $p_{*}(f(x))$ to $f(y))$.  

%Recall that $transport^{P}(p,_)$ is a function from a family of types to a family of types. We can specify the $u:P(x)$ that transport maps to by writing $transport^{P}(p,u)$.  

\begin{lem}\label{lem:2.3.5}
Fix some $A,B : U$ and b:B. If the type family $P:A \rightarrow U$ is defined to be $P(x):=B$  (not dependant on $A$).  Then there exists a path:
\[transportconst^{B}_{p}(b): transport^{P}(p,b) = b .\]
\end{lem}
 
\begin{proof}
We can assume that $p := refl_{x}$ so that we must consider $transport^{P}(refl_{x}.b) = b$.  By lemma \ref{lem:2.3.1}  $transport^{P}(refl_{x}.b):=b$ so we just need to construct a witness of type $refl_{b}$ and apply path induction.  
\end{proof}


Lemma \ref{lem:2.3.5} highlights the subtle difference between a \emph{function} and  a path: just compare $transport$ which is a function and tells us that a point $u$ goes to point $v$  and $transportconst$ which is a path between $u$ and $v$.  
So if $P$ is a type family independent from $A$ and given $x,y:A$ and a path between them $p:x=_{A}y$, then for any $b$ in the total space $B$ we have $transport^{P}(p,b)$ is a function from $P(x):=B$ to $P(y):=B$.  In particular, it sends $transport^{B}(p,b):B$ to $b:B$.  Furthermore $transportconst^{P}_{p}(b)$ is a \emph{path} between  $transport^{P}(p,b)$ and  $b$.  We will see that $transportconst$ makes sense when mapping \emph{within} a particular fibre $P(x)$.   

Now we wish to link the dependent and independent functions $ap_{f}(p)$ and $apd_{f}(p)$.

\begin{lem}\label{lem:2.3.6}
Let $f:A \rightarrow B$, then
\[apd_{f}(p) = transportconst_{p}^{B}(f(x)) \Box ap_{f}(p)\]
\end{lem}
%you need to define ap and concatenation
 \begin{proof}
By path induction it is enough to assume that $p$ is $refl_{x}$.  So we wish to construct a witness to type $apd_{f}(refl_{x}) = transportconst_{refl_{x}}^{B}(f(x)) \dot ap_{f}(refl_{x})$.  By lemma \ref{lem 2.3.5} $transportconst^{B}_{p}(b) = refl_{b}$, by lemma \ref{} $ap_{f}(refl_{x}) = refl_{f(x)}$  and by lemma \ref{} $refl_{x} \Box refl_{x} = refl_{x}$.   Putting this together we need to prove  $refl_{f(x)} = refl_{f(x)}$: we have $refl_{refl_{f(x)}}$ for this.  Path induction gives the required result.  
\end{proof}

In topological terms, given a fibration $\pi_{1}:\sum_{x:A}P(x)$ and a function $f:A \rightarrow B$ such that $f'(x) = (x,f(x))$ and a path $x = y$ in space $A$ we know there exists the following three maps in $B$:
\begin{itemize}
\item[(i)] $apd_{f}(p)$ is a path in the fibre, $P(y) := B$ over $y$ from $transport^{P}(p,f(x))$ to $f(y)$
\item[(ii)]$transportconst_{p}^{B}(f(x))$ is a path in $B$ from $transport^{P}(p,f(x)) :B$ to $f(x): B$.   
\item[(iii)]$ap_{f}(p)$ is the path from $f_{x}$ to $f(y)$ in $B$.  
\end{itemize}

Drawing these paths in $B$ assures us that Lemma \ref{lem:2.3.6} is well-typed.  
    






\end{document}

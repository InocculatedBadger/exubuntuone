% !TEX TS-program = pdflatex
% !TEX encoding = UTF-8 Unicode

% This is a simple template for a LaTeX document using the "article" class.
% See "book", "report", "letter" for other types of document.

\documentclass[12pt]{article} % use larger type; default would be 10pt
\usepackage{color}
\usepackage[utf8]{inputenc} % set input encoding (not needed with XeLaTeX)

%%% Examples of Article customizations
% These packages are optional, depending whether you want the features they provide.
% See the LaTeX Companion or other references for full information.

%%% PAGE DIMENSIONS

\usepackage[top=20mm,right=20mm,bottom=15mm,left=20mm]{geometry}
% \geometry{margins=2in} % for example, change the margins to 2 inches all round
% \geometry{landscape} % set up the page for landscape
%   read geometry.pdf for detailed page layout information

\usepackage{graphicx} % support the \includegraphics command and options

% \usepackage[parfill]{parskip} % Activate to begin paragraphs with an empty line rather than an indent

%%% PACKAGES
\usepackage{booktabs} % for much better looking tables
\usepackage{array} % for better arrays (eg matrices) in maths
%\usepackage{paralist} % very flexible & customisable lists (eg. enumerate/itemize, etc.)

%\usepackage{subfig} % make it possible to include more than one captioned figure/table in a single float
\usepackage{amsfonts}
\usepackage{amsthm}
\usepackage{tikz}
\usepackage{amsmath}
\usepackage{float}
\usepackage{graphicx}
\usepackage{caption}
\usepackage{subcaption}
\usepackage{color}
\usepackage{amssymb}
\usepackage{bm}

% These packages are all incorporated in the memoir class to one degree or another...

%%% HEADERS & FOOTERS
\usepackage{fancyhdr} % This should be set AFTER setting up the page geometry
\pagestyle{fancy} % options: empty , plain , fancy
\renewcommand{\headrulewidth}{0pt} % customise the layout...
\lhead{}\chead{}\rhead{}
\lfoot{}\cfoot{\thepage}\rfoot{}

%%% SECTION TITLE APPEARANCE
%\usepackage{sectsty}
%\allsectionsfont{\sffamily\mdseries\upshape} % (See the fntguide.pdf for font help)
% (This matches ConTeXt defaults)

%%% ToC (table of contents) APPEARANCE
%\usepackage[nottoc,notlof,notlot]{tocbibind} % Put the bibliography in the ToC
%\usepackage[titles,subfigure]{tocloft} % Alter the style of the Table of Contents
%\renewcommand{\cftsecfont}{\rmfamily\mdseries\upshape}
%\renewcommand{\cftsecpagefont}{\rmfamily\mdseries\upshape} % No bold!


\newtheorem{theorem}{Theorem} 
\newtheorem{lemma}{Lemma}
\newtheorem{propn}{Proposition}
\newtheorem*{thmm}{Theorem}
\newtheorem{remk}{Remark} 
\newtheorem{corol}{Corollary}
\newtheorem{definition}{Definition}



\newtheorem{thm}{Theorem}[section] 
\newtheorem{prop}[thm]{Proposition} 
\newtheorem{lem}[thm]{Lemma}
\newtheorem{cor}[thm]{Corollary} 
\newtheorem{con}[thm]{Conjecture} 

\theoremstyle{definition}
\newtheorem{defn}[thm]{Definition}
\newtheorem*{rem}{Remark}
%\newtheorem*{nota}{Notation}
\newtheorem*{nota}{Notation}
\newtheorem{cla}[thm]{Claim}
\newtheorem{ex}[thm]{Example}
\newtheorem{exs}[thm]{Examples}
\newtheorem*{exer}{Exercise}
\newtheorem{case}{Case}
\newtheorem{conj}{Conjecture}

\definecolor{sotonblue}{rgb}{0.0,0.394,0.597}


%opening
 \title{Preferential Attachment}
\author{David Matthews}

\begin{document}
 
 \section{Introduction}
 %degree distribution
 \subsection{Random Recursive Planar Trees}\label{RRPT}
 
 To build a random recursive planar tree,$T$, on $n$ nodes one begins with a root vertex and adds $n-1$ vertices one by one.  Let $V(T)$ be the vertex set of $T$. The descendants of any vertex $v \in V(T)$ are ordered (say from left to right), therefore if $v$ has outdegree $d$ there are $d+1$ distinct possible positions for a new vertex to attach to $v$.  So if the outdegrees of every vertex are $d_{1},d_{2},d_{3},\dots,d_{n}$ then the number, $N$, of possible attachment points for a new vertex is
 \begin{align*}
  N &= (d_{1} +1) + (d_{2} + 1) + \dots + (d_{n} + 1)\\
   &= n + (d_{1} + d_{2} + \dots d_{n}) \\
  &= n + (n-1) = 2n -1.
 \end{align*}
 Each new vertex is attached to a vertex of $T$ chosen uniformly at random from the $2n -1 $ possiblities.  To be even more explicit, given a random recursive planar tree $T$ on $n$ vertices then for chosen vertex $v$ with outdegree $d$ the probability, $p$ that the new vertex will be attached to $v$ is:
 \[p = \frac{d+1}{\sum_{i=1}^{n}d_{i}}.\] 
 
 \subsection{Barabasi-Albert Model}
 \subsection{Barabasi-Albert Model for Trees}
 
 Given a tree $T$ with vertices $V(T) = \{v_{1},v_{2},\dots,v_{n}\}$ such that the degree of each vertex is $deg(v_{i})$ for $i = 1,2,\dots n$ then we attach to node $v$ with probability proportional to $deg(v)$.  More specifically we choose node $v$ with probability $q$ where $q$ is defined as follows:
 \[q = \frac{deg(v)}{\sum_{v \in V(T)}deg(v)}.\]
 %sort out notation between the sections.
 
 Since $q$ defined above and $p$ from section \ref{RRPT} are equal there exists a corrospndence between the Barabasi-Albert prefferential attachment model for trees and random recursive planar trees.  Furthermore, the limiting behaviour of these trees must be the same.  
 
 
 
 \section{Degree Distribution}

Let $T$ be a random recursive planar tree on $n$ nodes and $X_{ni}$ be the number of vertices of outdegree $i \geq 0$ in a random plane recursive tree on $n$ vertices.

\begin{thm}
 Almost surely in the limit as $n \rightarrow \infty n^{-1}Y_{ni} \rightarrow \frac{4}{(i+1)(i+2)(i+3)}$.  
\end{thm}

 \begin{proof}
  See \cite{}, Theorem 1.3. 
 \end{proof}

 By the corrospondence between random recursive planar trees and Barabasi-Albert trees almost surely as $n \rightarrow \infty$ the degree distribution of a Barabasi-Albert tree tends to $\frac{4}{(i+1)(i+2)(i+3)}$.  
 
 \bibliographystyle{h-physrev3.bst}
\bibliography{./urns.bib}
 
\end{document}

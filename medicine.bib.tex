@article{Song,
  title={Origins of fractality in the growth of complex networks},
  author={Song, Chaoming and Havlin, Shlomo and Makse, Hern{\'a}n A},
  journal={Nature Physics},
  volume={2},
  number={4},
  pages={275--281},
  year={2006},
  publisher={Nature Publishing Group}
}

@article{Zamir,
author = {Zamir, M}, 
title = {The role of shear forces in arterial branching.},
volume = {67}, 
number = {2}, 
pages = {213-222}, 
year = {1976}, 
doi = {10.1085/jgp.67.2.213}, 
abstract ={A new optimality principle for the branching angles of blood vessels in the cardiovascular system is proposed: the principle of minimum drag. The results are examined in the light of general observations and compared with those obtained from the principles of minimum work and minimum volume. It is shown that in some aspects the new principle is equally consistent with observations, and, in other aspects, it is perhaps more plausible than the other two principles.}, 
URL = {http://jgp.rupress.org/content/67/2/213.abstract}, 
eprint = {http://jgp.rupress.org/content/67/2/213.full.pdf+html}, 
journal = {The Journal of General Physiology} 
}

@article{Cohn
year={1954},
issn={0007-4985},
journal={The bulletin of mathematical biophysics},
volume={16},
number={1},
doi={10.1007/BF02481813},
title={Optimal systems: I. The vascular system},
url={http://dx.doi.org/10.1007/BF02481813},
publisher={Kluwer Academic Publishers},
author={Cohn, DavidL.},
pages={59-74},
language={English}
}

@article {Cassot,
author = {CASSOT, FRANCIS and LAUWERS, FREDERIC and FOUARD, CÉLINE and PROHASKA, STEFFEN and LAUWERS-CANCES, VALERIE},
title = {A Novel Three-Dimensional Computer-Assisted Method for a Quantitative Study of Microvascular Networks of the Human Cerebral Cortex},
journal = {Microcirculation},
volume = {13},
number = {1},
publisher = {Blackwell Publishing Ltd},
issn = {1549-8719},
url = {http://dx.doi.org/10.1080/10739680500383407},
doi = {10.1080/10739680500383407},
pages = {1--18},
keywords = {cerebral microcirculation, confocal microscopy, human brain, morphometry, segmentation},
year = {2006},
}

@article{wellerperi,
  title={Perivascular Drainage of Amyloid-b Peptides from the Brain and Its Failure in Cerebral Amyloid Angiopathy and Alzheimer's Disease},
  author={Weller, Royo and Subash, Malavika and Preston, Stephend and Mazanti, Ingrid and Carare, Roxanao},
  journal={Brain Pathology},
  volume={18},
  number={2},
  pages={253--266},
  year={2008},
  publisher={Zurich, Switzerland: International Society of Neuropathology,[1990-}
}

@article{wellermicro,
  title={Microvasculature changes and cerebral amyloid angiopathy in Alzheimer’s disease and their potential impact on therapy},
  author={Weller, Roy O and Boche, Delphine and Nicoll, James AR},
  journal={Acta neuropathologica},
  volume={118},
  number={1},
  pages={87--102},
  year={2009},
  publisher={Springer}
}


@misc{Murray, 
title={The Physiological Principle of Minimum Work: I. The Vascular System and the Cost of Blood Volume.}, DOI={10.1073/pnas.12.3.207}, 
journal={Proceedings of the National Academy of Sciences of the United States of America}, 
author={C D Murray}, 
year={1926}}

@article{Preston,
          volume = {29},
          number = {2},
           title = {Capillary and arterial cerebral amyloid angiopathy in Alzheimer's disease: defining the perivascular route for the elimination of amyloid {\ensuremath{\beta}} from the human brain},
          author = {S.D. Preston and P.V. Steart and A. Wilkinson and J.A.R. Nicoll and R.O. Weller},
            year = {2003},
           pages = {106--117},
         journal = {Neuropathology \& Applied Neurobiology},
             url = {http://eprints.soton.ac.uk/27688/},
        abstract = {Accumulation of amyloid {\ensuremath{\beta}} (A{\ensuremath{\beta}}) in the extracellular spaces of the cerebral cortex and in blood vessel walls as cerebral amyloid angiopathy is a characteristic of Alzheimer's disease (AD) and the ageing human brain. Studies in animals suggest that A{\ensuremath{\beta}} is eliminated from the brain either directly into the blood or along perivascular interstitial fluid drainage channels. The aim of the present study is to define the perivascular route for the drainage of A{\ensuremath{\beta}} from the human brain. Smears and paraffin sections of post-mortem cortical tissue from 17 cases of AD and from two controls were stained with thioflavin and for A{\ensuremath{\beta}} by immunohistochemistry. Histology and confocal microscopy showed that deposits of A{\ensuremath{\beta}} in the cortical parenchyma were continuous with A{\ensuremath{\beta}} in capillary walls but A{\ensuremath{\beta}} in artery walls was not in continuity with A{\ensuremath{\beta}} in brain parenchyma. Quantitative studies supported these observations. The results of this study suggest that when A{\ensuremath{\beta}} is eliminated from the extracellular spaces of the human brain by the perivascular route, it enters pericapillary spaces and from there drains along the walls of cortical arteries to leptomeningeal arteries. Factors such as overproduction of A{\ensuremath{\beta}}, entrapment of A{\ensuremath{\beta}} in drainage pathways and poor drainage of A{\ensuremath{\beta}} due to functional changes in ageing arteries might result in the failure of elimination of A{\ensuremath{\beta}} from the ageing brain and play a major role in the pathogenesis of AD. Such factors might affect therapies for AD that entail administration of anti-A{\ensuremath{\beta}} antibodies to eliminate A{\ensuremath{\beta}} from the human brain.}
}

@book{bela,
  title={Modern graph theory},
  author={Bollob{\'a}s, B{\'e}la},
  volume={184},
  year={1998},
  publisher={Springer Verlag}
}

@article{Bengt,
  title={Alzheimer's disease: clinical trials and drug development},
  author={Mangialasche, Francesca and Solomon, Alina and Winblad, Bengt and Mecocci, Patrizia and Kivipelto, Miia and others},
  journal={Lancet neurology},
  volume={9},
  number={7},
  pages={702},
  year={2010}
}

@article{Rox,
          volume = {238},
          number = {4},
           month = {February},
          author = {D. Schley and R. Carare-Nnadi and C.P. Please and V.H. Perry and R.O. Weller},
           title = {Mechanisms to explain the reverse perivascular transport of solutes out of the brain},
         journal = {Journal of Theoretical Biology},
           pages = {962--974},
            year = {2006},
        keywords = {perivascular transport, thin-film flow, cerebral amyloid angiopathy, alzheimer's disease},
             url = {http://eprints.soton.ac.uk/60836/},
        abstract = {Experimental studies and observations in the human brain indicate that interstitial fluid and solutes, such as amyloid-{\ensuremath{\beta}} (A{\ensuremath{\beta}}), are eliminated from grey matter of the brain along pericapillary and periarterial pathways. It is unclear, however, what constitutes the motive force for such transport within blood vessel walls, which is in the opposite direction to blood flow. In this paper the potential for global pressure differences to achieve such transport are considered. A mathematical model is constructed in order to test the hypothesis that perivascular drainage of interstitial fluid and solutes out of brain tissue is driven by pulsations of the blood vessel walls. Here it is assumed that drainage occurs through a thin layer between astrocytes and endothelial cells or between smooth muscle cells. The model suggests that, during each pulse cycle, there are periods when fluid and solutes are driven along perivascular spaces in the reverse direction to the flow of blood. It is shown that successful drainage may depend upon some attachment of solutes to the lining of the perivascular space, in order to produce a valve-like effect, although an alternative without this requirement is also postulated. Reduction in pulse amplitude, as in ageing cerebral vessels, would prolong the attachment time, encourage precipitation of A{\ensuremath{\beta}} peptides in vessel walls, and impair elimination of A{\ensuremath{\beta}} from the brain. These factors may play a role in the pathogenesis of cerebral amyloid angiopathy and in the accumulation of A{\ensuremath{\beta}} in the brain in Alzheimer's disease.}
}



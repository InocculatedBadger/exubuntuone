% !TEX TS-program = pdflatex
% !TEX encoding = UTF-8 Unicode

% This is a simple template for a LaTeX document using the "article" class.
% See "book", "report", "letter" for other types of document.

\documentclass[12pt]{article} % use larger type; default would be 10pt
\usepackage{color}
\usepackage[utf8]{inputenc} % set input encoding (not needed with XeLaTeX)

%%% Examples of Article customizations
% These packages are optional, depending whether you want the features they provide.
% See the LaTeX Companion or other references for full information.

%%% PAGE DIMENSIONS

\usepackage[top=20mm,right=20mm,bottom=15mm,left=20mm]{geometry}
% \geometry{margins=2in} % for example, change the margins to 2 inches all round
% \geometry{landscape} % set up the page for landscape
%   read geometry.pdf for detailed page layout information

\usepackage{graphicx} % support the \includegraphics command and options

% \usepackage[parfill]{parskip} % Activate to begin paragraphs with an empty line rather than an indent

%%% PACKAGES
\usepackage{booktabs} % for much better looking tables
\usepackage{array} % for better arrays (eg matrices) in maths
%\usepackage{paralist} % very flexible & customisable lists (eg. enumerate/itemize, etc.)

%\usepackage{subfig} % make it possible to include more than one captioned figure/table in a single float
\usepackage{amsfonts}
\usepackage{amsthm}
\usepackage{tikz}
\usepackage{amsmath}
\usepackage{float}
\usepackage{graphicx}
\usepackage{caption}
\usepackage{subcaption}
\usepackage{color}
\usepackage{amssymb}
\usepackage{bm}

% These packages are all incorporated in the memoir class to one degree or another...

%%% HEADERS & FOOTERS
\usepackage{fancyhdr} % This should be set AFTER setting up the page geometry
\pagestyle{fancy} % options: empty , plain , fancy
\renewcommand{\headrulewidth}{0pt} % customise the layout...
\lhead{}\chead{}\rhead{}
\lfoot{}\cfoot{\thepage}\rfoot{}

%%% SECTION TITLE APPEARANCE
%\usepackage{sectsty}
%\allsectionsfont{\sffamily\mdseries\upshape} % (See the fntguide.pdf for font help)
% (This matches ConTeXt defaults)

%%% ToC (table of contents) APPEARANCE
%\usepackage[nottoc,notlof,notlot]{tocbibind} % Put the bibliography in the ToC
%\usepackage[titles,subfigure]{tocloft} % Alter the style of the Table of Contents
%\renewcommand{\cftsecfont}{\rmfamily\mdseries\upshape}
%\renewcommand{\cftsecpagefont}{\rmfamily\mdseries\upshape} % No bold!


\newtheorem{theorem}{Theorem} 
\newtheorem{lemma}{Lemma}
\newtheorem{propn}{Proposition}
\newtheorem*{thmm}{Theorem}
\newtheorem{remk}{Remark} 
\newtheorem{corol}{Corollary}
\newtheorem{definition}{Definition}



\newtheorem{thm}{Theorem}[section] 
\newtheorem{prop}[thm]{Proposition} 
\newtheorem{lem}[thm]{Lemma}
\newtheorem{cor}[thm]{Corollary} 
\newtheorem{con}[thm]{Conjecture} 

\theoremstyle{definition}
\newtheorem{defn}[thm]{Definition}
\newtheorem*{rem}{Remark}
%\newtheorem*{nota}{Notation}
\newtheorem*{nota}{Notation}
\newtheorem{cla}[thm]{Claim}
\newtheorem{ex}[thm]{Example}
\newtheorem{exs}[thm]{Examples}
\newtheorem*{exer}{Exercise}
\newtheorem{case}{Case}
\newtheorem{conj}{Conjecture}

\definecolor{sotonblue}{rgb}{0.0,0.394,0.597}


%opening
 \title{Question}
\author{David Matthews \\\\\emph{Supervisors: Dr James Anderson and Dr Ben MacArthur}}

\begin{document}
% \section{Local to Global properties of ``almost surely'' statements}
 
 Given some infinite collection of objects, $X$, if any $x \in X$ almost surely has property $\mathcal{P}$ what can we say about $X$?  To clarify what I mean consider interested in the following example.
 
 A \emph{random recursive tree} (RRT), $T_{n}$, with vertices $V(T) = \{v_{1},\dots,v_{n}\}$ is a labelled, rooted tree obtained by assigning a root vertex $v_{1}$ then adding $n-1$ vertices one by one such that each new vertex is joined by an edge to a randomly and uniformly chosen existing vertex.
 
 Local scenario:  Let $X_{ni}$ be the number of vertices of degree $i \geq 1$ in a RRT on $n$ vertices.  Janson [] proves that as $n \rightarrow \infty$ almost surely:
 \[\frac{X_{ni}}{n} \rightarrow 2^{-i}\]
 
 Global scenario:  in the limit as $n \rightarrow \infty$ then $\frac{Y_{ni}}{(n-1)!} \rightarrow ?$.  So my first hurdle is to appropriately define $Y_{ni}$.  Two possible versions of $Y_{ni}$:
 \begin{itemize}
  \item[(i)] $Y_{ni}$ is the number of RRTs on $n$ vertices such that $\frac{X_{ni}}{n} = 2^{-i}$ 
  \item[(ii)] $Y_{ni}$ is the number of RRTs on $n$ vertices that are in some appropriately defined neighbourhood (dependent on $n$ and possibly $i$) of $\frac{X_{ni}}{n} = 2^{-i}$. 
 \end{itemize}
I initially posed the question using definition (i), however this seems far to restrictive. 

Finally, I appear to be attempting to find an analogue of the expectation of an event, i.e. given that for each infinite tree $T$ we know that $T$ having the  degree distribution $2^{-i}$ is a measure 1 event but what proportion of trees have this property?      
 
\end{document}

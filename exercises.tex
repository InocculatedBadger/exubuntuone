% !TEX TS-program = pdflatex
% !TEX encoding = UTF-8 Unicode

% This is a simple template for a LaTeX document using the "article" class.
% See "book", "report", "letter" for other types of document.

\documentclass[12pt]{article} % use larger type; default would be 10pt
\usepackage{color}
\usepackage[utf8]{inputenc} % set input encoding (not needed with XeLaTeX)

%%% Examples of Article customizations
% These packages are optional, depending whether you want the features they provide.
% See the LaTeX Companion or other references for full information.

%%% PAGE DIMENSIONS

%\usepackage[top=20mm,right=20mm,bottom=15mm,left=20mm]{geometry}
% \geometry{margins=2in} % for example, change the margins to 2 inches all round
% \geometry{landscape} % set up the page for landscape
%   read geometry.pdf for detailed page layout information

\usepackage{graphicx} % support the \includegraphics command and options

% \usepackage[parfill]{parskip} % Activate to begin paragraphs with an empty line rather than an indent

%%% PACKAGES
\usepackage{booktabs} % for much better looking tables
\usepackage{array} % for better arrays (eg matrices) in maths
%\usepackage{paralist} % very flexible & customisable lists (eg. enumerate/itemize, etc.)

%\usepackage{subfig} % make it possible to include more than one captioned figure/table in a single float
\usepackage{amsfonts}
\usepackage{amsthm}
\usepackage{tikz}
\usepackage{amsmath}
\usepackage{float}
\usepackage{graphicx}
\usepackage{caption}
\usepackage{subcaption}
\usepackage{color}
\usepackage{amssymb}
\usepackage{bm}

% These packages are all incorporated in the memoir class to one degree or another...

%%% HEADERS & FOOTERS
\usepackage{fancyhdr} % This should be set AFTER setting up the page geometry
\pagestyle{fancy} % options: empty , plain , fancy
\renewcommand{\headrulewidth}{0pt} % customise the layout...
\lhead{}\chead{}\rhead{}
\lfoot{}\cfoot{\thepage}\rfoot{}

%%% SECTION TITLE APPEARANCE
%\usepackage{sectsty}
%\allsectionsfont{\sffamily\mdseries\upshape} % (See the fntguide.pdf for font help)
% (This matches ConTeXt defaults)

%%% ToC (table of contents) APPEARANCE
%\usepackage[nottoc,notlof,notlot]{tocbibind} % Put the bibliography in the ToC
%\usepackage[titles,subfigure]{tocloft} % Alter the style of the Table of Contents
%\renewcommand{\cftsecfont}{\rmfamily\mdseries\upshape}
%\renewcommand{\cftsecpagefont}{\rmfamily\mdseries\upshape} % No bold!


\newtheorem{theorem}{Theorem} 
\newtheorem{lemma}{Lemma}
\newtheorem{propn}{Proposition}
\newtheorem*{thmm}{Theorem}
\newtheorem{remk}{Remark} 
\newtheorem{corol}{Corollary}
\newtheorem{definition}{Definition}



\newtheorem{thm}{Theorem}[section] 
\newtheorem{prop}[thm]{Proposition} 
\newtheorem{lem}[thm]{Lemma}
\newtheorem{cor}[thm]{Corollary} 
\newtheorem{con}[thm]{Conjecture} 

\theoremstyle{definition}
\newtheorem{defn}[thm]{Definition}
\newtheorem*{rem}{Remark}
%\newtheorem*{nota}{Notation}
\newtheorem*{nota}{Notation}
\newtheorem{cla}[thm]{Claim}
\newtheorem{ex}[thm]{Example}
\newtheorem{exs}[thm]{Examples}
\newtheorem*{exer}{Exercise}
\newtheorem{case}{Case}
\newtheorem{conj}{Conjecture}

\definecolor{sotonblue}{rgb}{0.0,0.394,0.597}

\newcommand{\pspace}{$(\Omega_\alpha,\mathcal{F}_\alpha,P_\alpha)$ } 
\DeclareMathOperator{\Aut}{Aut}
\DeclareMathOperator{\Pspace}{(\Omega, \mathcal{F},\mathbb{P})}
\DeclareMathOperator{\Pspacen}{(\Omega_n, \mathcal{F}_n,\mathbb{P}_n)}

\DeclareMathOperator{\X}{\mathcal{X}}
\DeclareMathOperator{\Y}{\mathcal{Y}}
\DeclareMathOperator{\A}{\mathcal{A}}
\DeclareMathOperator{\B}{\mathcal{B}}
\DeclareMathOperator{\F}{\mathcal{F}}

%opening
 \title{Exercises}
\author{David Matthews}
































\begin{document}
 \begin{exer}
  Show that $\pi_0(BS^{-1}S^{iso}) = k_{0}(S^{iso})$.
 \end{exer}
\begin{exer}
 Let $B$ be a bilinear form $ B :  \mathbb{R}^n\times \mathbb{R}^n \rightarrow \mathbb{R}^n$ then $B$ can be described by an $n \times n$ matrix so that $u^T B v = B(u,v)$ Define:
 \[G_{B} = X \in Gl(n,\mathbb{R}) : B(Xu, Xv) = B(u,v) \text{  for all  } u,v, \in \mathbb{R}^n\}\] 
 Here $Gl(n,\mathbb{R})$ is the group of general linear matrices.  
\end{exer}

\begin{exer}
 Let $G$ be a group. Let $G$ act on itself by a map $\tilde{}$ and the stabiliser of an element $x \in G$ is written $t(x) = \{ g \in G : \tilde{g}(x) = x\}$.  
\end{exer}

\begin{exer}
 Show that a group $G$ acts transitively on Cosets. 
\end{exer}


 
 \begin{exer}
   State the Cellular approximation theorem.
 \end{exer}

 \begin{exer}
  What is the difference between singular and celluar homology?
 \end{exer}

 \begin{exer}
  What is a 1-form?
 \end{exer}

 \begin{exer}
  What is an excision?  (Morse theory).
  
 \end{exer}

 \begin{exer}
  A ring, $R$ is local iff for every $r \in R$ either $r$ is invertible or $1-r$ is invertible.
 \end{exer}

 \begin{exer}
  There exists a well defined homomorphism $f: \mathbb{Z}_p \rightarrow \mathbb{Z}p^d\mathbb{Z}$ where $f(a/b) = $
  Show that $ker(f) = p^d \mathbb{Z}_p$.  (Commutative algebra).
  \end{exer}

  \begin{exer}
   If $I$ is ideal in $R$ and every element of $R/I$ is invertible then $I$ is the only maximal ideal in $R$.
  \end{exer}

  \begin{exer}
   What is a CW module/ a space of CW type?
  \end{exer}

  \begin{exer}
   What is a homology theory?
  \end{exer}

  \begin{exer}
   What is Poincare duality?
  \end{exer}
\begin{exer}
 What is the smash product? Wedge product?   - see Hatcher

\end{exer}
\begin{exer}
 What is the precise dfinition of a suspension?
\end{exer}

\begin{exer}
 What is the profinite top?
 
\end{exer}

\begin{exer}
 What is a 1-form?
\end{exer}

  
\end{document}
